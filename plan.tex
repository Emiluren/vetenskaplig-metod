\documentclass[msc,lith,english]{liuthesis}
\usepackage[
    backend=biber,
    style=numeric,
  ]{biblatex}
\bibliography{sources}

\begin{document}
TDDD89 Thesis Proposal for group EMISE935, OLAOV121

Proposed thesis title

Analysing applicability and usefulness of Grover Search for common NP-hard problems.

\section{Problem Description}
\subsection{Background}
NP denotes the class of problems solvable in polynomial time given a non-deterministic turing machine. NP-hard problems are the hardest problems in NP, requiring super-polynomial time to execute on a deterministic computer. NP-hard will in this text also refer, largely though not exclusively, to NP-complete problems. NP-complete is the class of problems into which all of NP can be reduced in polynomial time. Reduce means to re-write as another problem using some algorithm.

Grover’s Search algorithm is a quantum algorithm for which solves SAT-problems (boolean constraint satisfaction) and related FNP (functional analogue to NP) problems with a quadratic reduction in complexity relative to trivial classical solutions. The most encompassing of SAT are NP-complete. The “search” phrasing relates to the applicability in ‘searching’ lists already converted into a quantum circuit, which it achieves in sub-linear time. While this polynomial speed-up is not enough to make NP-hard problems solvable in polynomial time, it may as much as double the computable input size from what is achievable in classical computing.

\subsection{Proposed thesis questions}
The proposed thesis tackles the following questions/tasks:
What are an appropriate set of example problems to analyse with regard to the following questions?

What, if any, NP-hard problems cannot be solved via reduction into an appropriate form followed by application of Grover’s search, such that a polynomial speed-up is achieved relative to classical algorithms?

Can a lower bound be found for these speed-ups?

If any problems are not successfully shown to be improved in this manner, can a lower bound for the reduction be produced, proving Grover’s Search cannot provide an improved solution?

Can any useful hypotheses be drawn from the findings, such as if specific subclasses of problems are particularly resistant? Given time, can these be effectively mapped onto other quantum algorithms/methods, such as quantum annealing?

\subsection{Purpose}
The benefit in knowing the answers to these questions are numerous. They chiefly pertain to improving calculability of otherwise difficult-to-approach problems by potentially as much as doubling the input sizes for which they are solvable, given future advances in construction of quantum computers; potentially the only such option if both BQP (quantum-computer polynomial) and N (deterministic turing machine-polynomial) complexity classes are separate from NP. There is also a hope to better the understanding of interrelations between problems, particularly within the NP-complete set. Lastly, the project aims to increase interest in quantum computation and algorithmics, by presenting uses outside of solving BQP, should a delimitation on the class ever be shown.

\section{Initial Approach}
Firstly, a delimitation to a reasonable number of target problems must be established, proposedly via discussion with the assigned supervisor. A selection must then be made as to which problems specifically to target. The initial approach will be to aim for a set of representative problems first and foremost. Inclusion of different variations and relaxations of the same problem, such as testing for finding both hamilton paths and cycles, will be avoided. Only a particular class or type of problem tested should yield particular results, further problems within the class may be tested as well. For potentially proving any lower bounds, we will attempt whatever approaches appear applicable to the relevant reduction, to show a growth in problem input size by less than a factor of 2, at which point the Grover Search algorithm no longer improves the calculability of the problem. It is worth noting that any NP-hard problem approached will be solvable by the proposed method, since all problems in NP can be reduced to any NP-complete problem, thus also to SAT-problems to which Grover’s algorithm can be applied. Furthermore, no actual quantum computation or simulation need be performed for the sake of the thesis; the work to be done is bounded to classical problem reduction and analysis.

\end{document}